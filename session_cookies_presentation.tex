\documentclass{beamer}
\usetheme{Madrid} % hoặc thử các theme khác như: Berlin, Warsaw, CambridgeUS

\title{Session và Cookies trong Web Development}
\author{Nguyen Khai}
\date{\today}

\begin{document}

\begin{frame}
  \titlepage
\end{frame}

% --- Slide: Tổng quan ---
\begin{frame}{Tổng quan}
  \begin{itemize}
    \item Web hoạt động theo mô hình stateless – mỗi request là độc lập.
    \item Để duy trì thông tin người dùng giữa các request, ta dùng:
    \begin{itemize}
      \item \textbf{Cookies}
      \item \textbf{Sessions}
    \end{itemize}
  \end{itemize}
\end{frame}

% --- Slide: Cookies là gì ---
\begin{frame}{Cookies là gì?}
  \begin{itemize}
    \item Là dữ liệu nhỏ được lưu trên trình duyệt người dùng.
    \item Mỗi lần gửi request, cookies sẽ được gửi kèm theo.
    \item Dùng để lưu: thông tin đăng nhập, tùy chọn ngôn ngữ, token, v.v.
  \end{itemize}
\end{frame}

% --- Slide: Session là gì ---
\begin{frame}{Session là gì?}
  \begin{itemize}
    \item Là bộ nhớ lưu trữ thông tin người dùng trên server.
    \item Khi user đăng nhập, server tạo một session và gắn session ID.
    \item Session ID thường được lưu dưới dạng cookie.
    \item Dùng để lưu: user ID, role, giỏ hàng, trạng thái đăng nhập, v.v.
  \end{itemize}
\end{frame}

% --- Slide: So sánh ---
\begin{frame}{So sánh Session và Cookies}
  \begin{tabular}{|l|l|l|}
    \hline
    \textbf{Tiêu chí} & \textbf{Cookies} & \textbf{Sessions} \\
    \hline
    Lưu trữ & Trình duyệt (client) & Server \\
    \hline
    Dung lượng & ~4KB & Tuỳ ý (RAM/DB) \\
    \hline
    Bảo mật & Kém hơn & Bảo mật hơn \\
    \hline
    Thời gian sống & Có thể rất lâu & Thường ngắn, timeout \\
    \hline
  \end{tabular}
\end{frame}

% --- Slide: Minh họa ---
\begin{frame}{Minh họa hoạt động}
  \begin{enumerate}
    \item User truy cập website → Gửi request đến server.
    \item Server tạo session → lưu user info → gửi session ID về trình duyệt.
    \item Trình duyệt lưu session ID trong cookie.
    \item Các lần request sau → gửi lại cookie → server tra session.
  \end{enumerate}
\end{frame}

% --- Slide: Ứng dụng ---
\begin{frame}{Ứng dụng thực tế}
  \begin{itemize}
    \item Xây dựng hệ thống đăng nhập người dùng.
    \item Giỏ hàng thương mại điện tử.
    \item Ghi nhớ tùy chọn người dùng (theme, ngôn ngữ).
  \end{itemize}
\end{frame}

% --- Slide: Lưu ý bảo mật ---
\begin{frame}{Lưu ý bảo mật}
  \begin{itemize}
    \item Cookies dễ bị đánh cắp qua XSS → nên mã hóa token.
    \item Nên dùng HTTPS để truyền dữ liệu.
    \item Session nên có timeout để tránh lạm dụng.
    \item Xóa session/cookie khi logout.
  \end{itemize}
\end{frame}

% --- Slide: Kết luận ---
\begin{frame}{Kết luận}
  \begin{itemize}
    \item Cookies và Sessions giúp duy trì trạng thái người dùng.
    \item Cần chọn cơ chế phù hợp và đảm bảo bảo mật.
    \item Kết hợp cả hai để đạt hiệu quả cao nhất.
  \end{itemize}
\end{frame}

\begin{frame}
  \centering
  \Huge Cảm ơn!
\end{frame}

\end{document}
